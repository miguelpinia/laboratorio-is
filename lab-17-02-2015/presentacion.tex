\documentclass{beamer}
\usepackage[utf8]{inputenc}
\usepackage[spanish]{babel}
\usetheme{Goettingen}
\usecolortheme{default}
\useoutertheme{shadow}
\useinnertheme{rectangles}
\graphicspath{ {./figures/} }
\title[Sesiones]{Sesiones con Java}
\subtitle{Ingeniería de Software}
\author[Miguel]{Miguel Angel Piña Avelino}
\institute[UNAM]{
  Facultad de Ciencias, UNAM
}
\date{\today}

\begin{document}

\frame{\titlepage}

\begin{frame}
  \frametitle{Índice}
  \tableofcontents
\end{frame}

\section{Introducción}
\begin{frame}
  \frametitle{Introducción}
  Hoy vamos a comenzar a trabajar con sesiones.\\
  El término sesión se refiere a una serie de interacciones del usuario con la
  aplicación que son rastreados por el servidor. Las sesiones son usadas para
  mantener un estado específico del usuario, incluyendo objetos persistentes
  e identidades de usuario autenticadas.
\end{frame}

\begin{frame}
  \frametitle{¿Cómo usar sesiones?}
  Para crear un objeto de tipo Session, vamos a hacer uso de HTTPServletRequest,
  a través del método getSession().\\
  Una vez establecida la sesión, examinaremos y usaremos sus propiedades con una
  pequeña aplicación.
\end{frame}

\begin{frame}[fragile]
  \frametitle{Creando o accediendo a una sesión}
  Para crear una nueva sesión o obtener acceso a una existente, hacemos lo
  siguiente:
  \begin{verbatim}
    HttpSession mySession = request.getSession();
  \end{verbatim}
  El método getSession() regresa un objeto de sesión válido asociado con el
  request. Si utilizamos el método con un parámetro booleano verdadero, este
  crea la sesión, si usamos falso, no la crea.
\end{frame}

\begin{frame}[fragile]
  \frametitle{Ejemplo}
  \begin{verbatim}
    public void doPost (HttpServletRequest req
                        , HttpServletResponse res)
    throws ServletException, IOException{
        if ( HttpSession session
                = req.getSession(false) ) {
           // Sesión recuperada, continua con
          // las operaciones del servlet
        }
        else{
           // No hay sesión, regresamos un mensaje de error.
        }
    }
  \end{verbatim}
\end{frame}

\end{document}

%%% Local Variables:
%%% mode: latex
%%% TeX-master: t
%%% End:
