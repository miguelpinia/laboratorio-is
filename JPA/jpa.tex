% Created 2015-03-02 lun 08:08
\documentclass[11pt]{article}
\usepackage[utf8]{inputenc}
\usepackage[T1]{fontenc}
\usepackage{fixltx2e}
\usepackage{graphicx}
\usepackage{longtable}
\usepackage{float}
\usepackage{wrapfig}
\usepackage{rotating}
\usepackage[normalem]{ulem}
\usepackage{amsmath}
\usepackage{textcomp}
\usepackage{marvosym}
\usepackage{wasysym}
\usepackage{amssymb}
\usepackage{hyperref}
\tolerance=1000
\author{Miguel Angel Piña Avelino}
\date{\today}
\title{Introducción a JPA}
\hypersetup{
  pdfkeywords={},
  pdfsubject={},
  pdfcreator={Emacs 24.3.1 (Org mode 8.2.10)}}
\begin{document}

\maketitle
\tableofcontents


\section{¿Qué es JPA?}
\label{sec-1}
Java Persistence API (JPA) es una herramienta que provee a los desarrolladores
java facilidades para realizar mapeos entre relaciones y objetos (object/relational
mapping) en aplicaciones Java.

\section{¿Donde usar JPA?}
\label{sec-2}

Para reducir la carga de escribir código para gestiónar la interacción entre la
base de datos y el código en Java, el trabajo relacional de objetos, un
programador usa el framework JPA, lo que permite una fácil interacción con
instancias de base de datos y las aplicaciones en Java.

\href{./jpa_provider.png}{Flujo de trabajo con JPA.}
% Emacs 24.3.1 (Org mode 8.2.10)
\end{document}